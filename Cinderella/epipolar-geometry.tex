<style>
p {
  text-indent: 20px;
}
li {
  text-indent: 20px;
}
</style>

<h1>Epipolar Geometry</h1> 
 
The animation (generated with <a
    href="http://www.cinderella.de">Cinderella</a>) 
    shows the main elements of the epipolar geometry of an image pair:
    the basis O'O'' between the two projection centres O' and O'' (red) of two cameras looking downwards, the epipoles E' and E'' (green) being the image of the other projection centre, the 3D-point P, its images P' and P'' (green), the epipolar lines e' and e'' (green), being the images of the projection rays O'P and O''P and the epipolar plane (green) containing all these elements.

    <p> You may change the parameters of the configuration (principle distance/focal length c, length B<sub>X</sub> of the basis, heights  Z<sub>1</sub> and Z<sub>2</sub> of the projection centres) by the sliders on the left. 
    You may change the image size by moving the right point in the image in the left figure. The 3D point is given by the  position P' which may be chosen freely and by the depth Z changable by the slider - yielding the position in the axonometric view.
    </p>

    <p> The axonometric view can be shifted using the point Y. The axonometric view may be rotated using the red point within (!) the circle on the lower left. </p>

    <p> Do not choose a view precisely parallel to a coordinate axis as sngularities may occur. In case the drawing is lost you may restore the graphics by reloading the applet. Some areas are indicated in color, however as the system does not know anything about the 3D geoemtry occlusions may appear incorrect  </p>


    <p> The red points are changable influencing only the corresponding parameter. The other parameters remain unchanged</p>



    <h2>Explore the geoemtry of the image pair</h2>
    <ul>

      <li> - Confirm yourself by moving P' that all elements of the epipolar geometry lie in the epipolar plane.


	<li> - Search for a configuration where the epipolar lines in one image are nearly parallel. Characterize this configuration. 

	  <li> - Search for a configuration where the epipolar lines pass through the principle point (here the centre of the image). Characterize this configuration. 

	    <li>  - Search for a configuration where the light area in the left image exactly correponds to the light area in the right image. How does this configuration depend on the depth of P?

	      <li> - Investigate the situation where the principle distance is negative. Which areas in the two images correspond?

    </ul>