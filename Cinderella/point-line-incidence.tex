<style>
p {
  text-indent: 20px;
}
li {
  text-indent: 20px;
}
</style>

<h1>Point-Line Incidence</h1> 


The  animation (generated with  <a href="http://www.cinderella.de">Cinderella</a>) shows ways to check the incidence of a point and a line.
    
<p> The check is based on the distance d<sub>xl</sub> of the point x and the line l. Usually, a geometric viewpoint compares the distance with some critical value, say c, which heuristically is related to the expected uncertainty of the point and the line. </p>

<p> In case the uncertainty of the point and the line parameters are known, say, from some previous experiments and can be assumed to be normally distributed, then the distance d<sub>xl</sub> - up to linearisation - also is normally distributed with some standard deviation &sigma;<sub>xl</sub> leading to a test statistic z=d<sub>xl</sub>/&sigma;<sub>xl</sub>, which, in case the incidence holds, is normally distributed according to N(0,1).
</p>

<p> The standard deviation &sigma;<sub>xl</sub>, however, turns out to depend on the relative position of the point and the line and the, possibly anisotropic, uncertainty of the point. Points of the line generally are uncertain, due to both, the uncertainty of the line across the line and the uncertainty of its direction with some point x<sub>O</sub> on the line with minimal uncertainty across the line. This can be visualized by two branches of a hyperbola, which is the set of all ends of standard intervals across the line. The uncertainty of the point can be visualized by the standard ellipse, providing the standard deviations a and b in two mutually orthogonal directions.  </p>

<p> In the animation the line is assumed to be generated by two uncertain points A and B with isotropic uncetainty, i.e. standard deviations &sigma;<sub>A</sub> and &sigma;<sub>B</sub>, leading to the line (AB) with distance d to the origin. You can interactively change the positions and the standard deviations of the two points. In addition, the point C can be chosen freely, together with the principle axes a and b of its standard ellipsoid and the direction of the principle axis, by rotating the blue line. In addition you can choose some critical value c for geometrically evaluating, whether the point C is close to (AB) using the green parallel lines.</p> 

    
<p> Depending on the chosen configuration, the animation shows the test statistic z on the lower left. </p>
    

    <h2>Explore the configuration:</h2>
    <ul>
      <li> - Choose visibly different values for a and b and rotate the standard ellipse of the point, without changing ist position and observe the effect onto the test statistic. </li>
 			
 			<li> - Choose a fixed  uncertainty of the point and move the point along one of the green lines, which indicate the distance c of the line (AB). Observe, the change of the test statistic. </li>
 			
 			<li> - Explore a trajectory of the point such that the test statistic is constant, say z=1.5 or z=3.</li>
 			
	    <li> - Explore the change of the uncertainty of the line (AB) as a function of the two standard deviations &sigma;<sub>A</sub> and &sigma;<sub>B</sub>, the distance of the two points A and B, and the direction of the line.</li>
	        </ul>