<style>
p {
  text-indent: 20px;
}
li {
  text-indent: 20px;
}
</style>


<h1>Decision Boundary for Gaussian Distribution</h1> 


The animation (generated with <a
    href="http://www.cinderella.de">Cinderella</a>) shows the decision boundary for the two-class problem with normally distributed features.
    
<p>    The two distributions (black and blue) are represented by three iso-lines of the probability density functions, being con-central similar ellipses. We assume the iso-lines represent the same density for the two classes. Thus, the decision boundary must pass through the intersection points of corresponding iso-lines. This decision boundary generally is a conic. </p>
    

<p>    The initial configuration shows two Gaussian distribution which have the same covariance matrix. The resulting decision boundary is a straight line.</p>    
    
<p>    You may change the two distributions by moving the centres (means), changing the semi-axes of the ellipses (left) and by rotating one of the axes. The animation then shows the resulting decision boundary, wich generally is a conic. </p>

<p>   In addition the Bhattacharyya-distance with its translation component (the Mahalanobis distance) and its covariance component, and the Hellinger distance are given.</p>
    
    
    

    <h2>Explore the configuration:</h2>
    <ul>

      <li> - Move the means of the two distributions and observe the change of the change of the 
      decision boundary. Can you give a simple rule for the position of this decision line?
 			
 			<li> - Change both covariances matrices such that the iso-lines are circular and of the 
 			same size. Again, can you give a simple rule for the position of this decision line?
 			
 			<li> - Change the mean and covariance of both distributions such that the decision 
 			boundary is an ellipse/hyperbola?  
 			
	    <li> - Explore under which condition the decision boundary is an ellipse and the two means 
	    are close together?
	    
	    <li> - Explore under which condition the two classes can be distinguished if the mean 
	    values are the same?
	    
	   
    </ul>
