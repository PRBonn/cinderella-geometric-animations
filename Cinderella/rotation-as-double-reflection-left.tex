<style>
p {
  text-indent: 20px;
}
li {
  text-indent: 20px;
}
</style>

<h1>Rotation as Double Reflection</h1> 


The animation (generated with <a
    href="http://www.cinderella.de">Cinderella</a>) visualizes that a rotation can be represented as a double reflection.
    
<p>  The figure can be interpreted in 2D and 3D. In 2D it shows the rotation of the quadrangle (ABCD) around O. It also is a double reflection at the two lines n (white) and m (red).  In 3D the drawing is meant to show the rotation around the rotation axis, which passes through O and is perpendicular to the drawing plane. The quadrangles may have arbitrary depths along the rotation axis. The 3D rotation also is a double reflection, now, at two planes perpendicular to the drawing planes, with normals n and m. </p>
    

<p>    The form of the quadrangle can be changed by moving the four points A, B, C, and D. The first mirror line/plane n (red) can be rotated around O. The angle &alpha; between the second mirror line/plane m (white) can be changed by rotating the red point on the circle. The mirrored quadrangles are shown together with the rotation angle &beta;, which is double the angle between the planes.</p>    
    

    
  

    <h2>Explore the configuration:</h2>
    <ul>

      <li> - Geometrically verify that the motion from (ABCD) to (A''B''C''D'') is a rotation, i.e. all distances to the the rotation centre O are preserved, and the angle of rotation for all points is the same.</li>
 			
 			<li> - Change the angle &alpha; of rotation and observe the motion of the quadrangles. When is the motion A to A'' a mirroring at the point O? For which angle(s) &alpha; is the rotation the identity?  </li>
 			
 			<li> - Change the orientation of the first mirror n, and observe the effect onto the configuration. Obviously, the mirroring A to A'' is invariant to the choice of n. </li>
 			
	    
	   
    </ul>