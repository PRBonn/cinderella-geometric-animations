<style>
p {
  text-indent: 20px;
}
li {
  text-indent: 20px;
}
</style>

h1>The Depth of Field of a Camera</h1> 
  
 The animation (generated with <a
    href="http://www.cinderella.de">Cinderella</a>) 
    allows to investigate the effect of camera parameters onto the depth of field, i.e. the range where the object is imaged sharply. 
    
    <p>Generally, only objects in a certain depth range lead to a sharp projection on the sensor of a camera. Objects farther away or closer to the camera lead to blurry images. Photographers exploit this effect, what is called the Bokeh effect: They place the object of interest into the depth of field and such that the object is shown sharply and the background appears blurred.</p>
    
    
    
    <p>In the animation you can change several parameters of a lens (certainly not covering all types of cameras and scenes), which are shown as red circles.  (1) the focal length f, (2) the aperture a by setting the f-stop f/a, (3) the relative pose of the object w.r.t. the camera either by moving the object (left) or the camera (F), and (4) the maximum circle of confusion, i.e. the maximum allowable blur for achieving a sharp image, and - just for good visualization the height of the object and half the vertical diameter d of the sensor. The depth of field is the light green area around the scene point. It is bounded by the near and the far limit. The difference of both is the depth of field. </p>
     
    
    
    
    
    <h3>Explore the mutual relation of the different characteristics:</h3>
    <ul>
    
    <li> - Change each parameter individually and observe the effect onto the configuration and the depth of field. 
		</li>
       
    <li> - Assume you want to realize the Bokeh effect when photographing a flower. Assume you want to have a distance of appr. 12 cm to the flower, whose 3D structure requires, that you have a depth of field of 2 cm. Find a configuration such that the flower is imaged sharply and - consequently the objects behind of and in front of the flower are shown blurry.</li>
    
    <li> - Cameras usually do not allow to take images which are too close to the camera. Assume a camera without zoom possibilities. While the focus point F has a fixed distance to the centre of the lens, in real cameras the sensor plane (white) can be adapted to achieve sharp images. Explain, why there is a technical limit which does not allow to take images of very close objects.</li>
     
    </ul>
