<style>
p {
  text-indent: 20px;
}
li {
  text-indent: 20px;
}
</style>

h1>Cross Ratio</h1> 
The animation (generated with <a href="http://www.cinderella.de">Cinderella</a>) visualizes the invariance of the cross ratio of four points under perspective transformations. 
<p> Taking the image of a straight line changes the distance ratios of collinear points, say the ratio R(A,B|C)=(a-c)/(b-c) of the coordinates of three points A,B, and C. However, four colinear points possess an invariant, namely the cross-ratio CR. For four points A, B, C, and D, it can be defined as a ratio of two ratios, namely CR(A,B|C,D)=R(A,B|C)/R(A,B|D). The cross ratio of four concurrent lines with directions &alpha;, &beta;, &gamma;, and &delta; can be defined similarly, replacing the coordinate differences by the sines of the direction differences, e.g. replacing  a-b by sin(&alpha;-&beta;).
</p>
<p> In the animation one can move the (red) points A, E, and F, rotate the (red) lines (AD) and (GH) and move the points B, C, D on the line (AD). The four points (ABCD)  are perspectivly mapped from F leading to the four points (GKLH) on the line (GH). The figure shows the corresponding distances among the points of each quadruple and the cross ratio of the configuration. All quadruples of collinear points and concurrent lines have the same cross ratio.</p>
    <h2>Explore the configuration:</h2>
    <ol>
      <li>  - Choose a configuration where the cross ratio is 1, 0.5, -0.5 and -2.</li>
			
      <li>  - What does the sign of the cross ratio indicate?</li>
			
			<li>  - Four points lie harmonic, if their cross ratio is -1. Choose a configuration where CR=-1. What does this mean for the distances AC and AD w.r.t. AB?</li>
 			
 			<li>  - Choose a harmonic point quadruple with D outside the interval AB. Rotate the line GL such that K is in the middle of GH. Where does the point L lie? This geometric configuration is useful for finding a point on the horizon when observing a straight road with markers of the middle line, see text.</li>
			
		
	    
	        </ol>
