<style>
p {
  text-indent: 20px;
}
li {
  text-indent: 20px;
}
</style>

<h1> Scaled Rotation: Geometry of the Product of two Complex Numbers </h1>

The animation, generated with <a href="http://www.cinderella.de">Cinderella</a>, allows to explore the multiplication of two complex numbers and interpret this operation geometrically as scaled rotation<br> 

<p>You can specify the real and the imaginary parts of the two complex numbers z<sub>1</sub>=x<sub>1</sub> + iy<sub>1</sub> and z<sub>2</sub>=x<sub>2</sub> + iy<sub>2</sub> by moving the yellow and blue points on the left, resp. The two points z<sub>1</sub> and  z<sub>2</sub> initially are shown in two different coordinate systems, namely  at O and O'. To both points there is an associated triangle including the origin z=0 and the point z=1, namely (O1z<sub>1</sub>) and (O'1'z<sub>2</sub>). 
</p>

<p> We now can realize a scaled rotation of the blue triangle around the origin O. For this, the blue triangle is copied to (QRz): (1) First, shift the blue triangle such that Q lies as O, then the third points of the two triangles represent the complex numbers in the same coordinate system. (2) Now, rotate the blue triangle around O, such that the red line segment falls onto the line (O<sub>1</sub>). This is a rotation by the angle phi_1= (z<sub>1</sub>,O,1). (3) Finally, the scaling of the triangle can be realized by moving the point P<sub>0</sub> on the left such that the point z of the blue triangle lies above z<sub>1</sub>. This is a scaling of the blue triangle by the length |z<sub>1</sub>|. If these operations are successful, the rotated and scaled point represents the product z= x + i y = z<sub>1</sub>z<sub>2</sub> = (x<sub>1</sub> x<sub>2</sub> -y<sub>1</sub> y<sub>2</sub>) + i (x<sub>1</sub> y<sub>2</sub> + x<sub>2</sub> y<sub>1</sub>) of the two given complex numbers.</p><br>

<h3>Explore the animation. </h3>	
		
		<ul>
		 <li> --- You now may change the coordinates of point z<sub>2</sub> and observe how their product z changes <br>
		  Observe: The transformation of the original small blue triangle (right) to the (now) larger blue triangle (left) is a planar similarity, since it includes the translation from O' to O. </li>
		<li> If you change the coordinates of point z<sub>2</sub> you need to update the scaling and the rotation </li>
		
		  <li> --- Can you choose two points, none of them being z = 1, such that the product is z = 4i? </li>
		  <li> --- Can you choose two points, none of them being z = 1, such that the product is real? </li>
		  <li> --- Can you choose two identical points, such that the product is z = -4? </li>
		</ul>
